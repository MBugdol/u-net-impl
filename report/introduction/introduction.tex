\documentclass[../main.tex]{subfiles}

\begin{document}

\newcommand{\colorsqr}[1]{
	\fcolorbox{black}{#1}{\rule{0pt}{6pt}\rule{6pt}{0pt}}
}

The goal of this project was to train an aritifical neural network to recognize camouflaged animals in images.
The network is a semantic segmentation model, which takes an image and classifies each pixel of the image into one of the following categories:
\begin{itemize}
	\item \textbf{masking background} (green \colorsqr{green}): the sections made by pixels of this category constitute the object the animal is camouflaging in,
	\item \textbf{animal} (blue \colorsqr{blue}): the sections made by these pixels represent the hidden animal,
	\item \textbf{attention} (white \colorsqr{white}): the sections made by pixels of this class draw attention (due to their size, shape, color...), even though they are neither part of the animal nor the camouflaging classes.
	      An example would be a colorful flower in an image of a fox hiding in the snow,
	\item \textbf{background} (red \colorsqr{red}): the pixel doesn't belong to any other category.
\end{itemize}
The model output is then taken and processed to create a mask image according to the above color-codes.

\end{document}